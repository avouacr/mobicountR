\documentclass[12pt]{article}
\usepackage[margin=1in]{geometry}
\usepackage[T1]{fontenc}
\usepackage[utf8]{inputenc}
\usepackage{amsmath,amssymb}


\linespread{1.25}
\setlength{\parindent}{0pt}

\begin{document}

\begin{center}
\LARGE \textbf{Description des stages longs}
\end{center}

\section{Stage à la DSDS de l'INSEE, 11/06/18 - 31/08/18}

\subsection{Contexte}

Ce stage a été réalisé à la section Revenu des Ménages de la Direction des Statistiques Démographiques et Sociales de l'INSEE. Ce stage était l'occasion d'acquérir une expérience de recherche en sociologie quantitative. La thématique de recherche a été proposé par Louis-André Vallet, et la supervision du stage a été réalisée conjointement par Céline Goffette (CREST) et Jérôme Accardo (INSEE).

\subsection{Problème de recherche}

Pour un certain nombre de pays et particulièrement dans le cas de la France, les conclusions des différentes littératures sur la transmission du statut socio-économique ne sont donc pas convergentes : là où les sociologues mettent en évidence une hausse de la fluidité sociale, i.e. un affaiblissement du lien entre origine et position sociales, les économistes mettent au contraire en évidence une transmission substantielle de la capacité à générer des revenus qui semble se perpétuer au fil des générations. L'objectif de cette étude est de proposer une analyse intermédiaire entre les approches sociologique et économique de la mobilité en étudiant dans quelle mesure la capacité à générer des revenus se transmet au fil des générations selon l'origine sociale des individus.

\subsection{Méthodologie}

Cette étude a pour objet la comparaison des distributions de revenus conditionnelles à l'origine sociale. Nous nous inscrivons dans la lignée du cadre économique et de la procédure statistique proposés par Lefranc, Pistolesi et Trannoy (2004) pour analyser l'évolution de l'inégalité des chances entre 1979 et 2000. Les auteurs proposent de définir l'inégalité des chances en recourant à une expérience de pensée : supposons que les individus ont la possibilité de choisir leur milieu social d'origine. L'inégalité des chances prévaut dès lors qu'un individu rationnel préfère toujours une distribution de revenus à une autre. Et comme le choix hypothétique de l'individu s'apparente à un choix risqué -- on compare des distributions de probabilités sur les différents niveaux de revenu -- la comparaison des distributions se fait à partir d'un critère de dominance stochastique.

\subsection{Données}

Nous appliquons ce cadre statistique à la série des enquêtes Revenus Fiscaux (ERF) et Revenus Fiscaux et Sociaux (ERFS) de l'Insee. Ces enquêtes couvrent la période 1996-2015, et permettent en cela de déterminer si la réduction de l'inégalité des chances mise en évidence par les auteurs entre 1979 et 2000 s'observe également au cours des deux dernières décennies.

\subsection{Résultats}

Nous présentons tout d'abord des statistiques descriptives qui mettent en évidence une inégalité des chances substantielle au niveau statique. Puis nous exposons les résultats issus de l'application de la procédure de Lefranc et al aux ERFS, qui met en lumière une stabilité de l'inégalité des chances entre 1996 et 2015, contrastant avec la forte réduction de l'inégalité à laquelle concluent les auteurs pour la période 1979-2000.

\section{Stage à Orange, 11/03/19 - 24/05/19}

\subsection{Contexte}

Ce stage a été ralisé au laboratoire SENSE d'Orange Labs. Il constitue le prolongement de mon stage d'application réalisé conjointement à l'INSEE et à Orange Labs. La raison de ce prolongement a été la possibilité de travailler sur des données mobiles très récentes, permettant d'envisager des estimations de population présente de haute précision spatio-temporelle. Comme le stage d'application, il a été réalisé conjointement sous la supervision de Benjamin Sakarovitch, \textit{data scientist} au SSP-Lab de l'INSEE, et de Zbigniew Smoreda, sociologue au laboratoire SENSE d'Orange.

\subsection{Problème de recherche}

Le problème de recherche est identique à celui du stage d'application : la localisation géographique des évènements. Cependant, contrairement à l'étude effectuée lors du stage précédent, l'enjeu est ici de s'affranchir complètement de la modélisation par polygônes de Voronoï dans le cadre du \textit{mapping} spatial des évènements. Cette extension est rendue possible par la disponibilité de données précises sur la couverture théorique des antennes.

\subsection{Méthodologie}



\subsection{Données}

\textit{signalling} (i.e. non plus seulement des données actives -- appels et SMS -- mais également passives -- connexions fréquentes du mobile aux antennes sans action de l'usager).

\subsection{Résultats}

\section{Stage à Foundamental (Berlin), 11/06/19 - 06/09/19}

\subsection{Contexte}

\subsection{Problème de recherche}

\subsection{Méthodologie}

\subsection{Données}

\subsection{Résultats}


\end{document}
